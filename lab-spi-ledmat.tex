\section{صفحه نمایش ماتریسی}

\subsection{اهداف آزمایش}
\begin{itemize}
    \item آشنایی با پروتوکل \lr{SPI}
    \item آشنایی با صفحه نمایش ماتریسی
\end{itemize}
\subsection{قطعات مورد نیاز}

\begin{itemize}
    \item بورد \lr{Arduino Uno}
    \item صفحه نمایش ماتریسی به همراه چیپ \lr{MAX7219}
\end{itemize}
\subsection{مقدمه}

در این آزمایش می‌خواهیم با با استفاده از یک صفحه نمایش ماتریسی، یک متن را به سبک تابلوهای روان تبلیغاتی نمایش دهیم. ابتدا باید با مفاهیم پایه‌ای مربوط به صفحه‌نمایش ماتریسی و چیپ \lr{MAX7219} آشنا شویم.

\newline
\begin{nas} صفحه نمایش ماتریسی\end{nas}
\newline

به طور عادی برای کنترل یک صفحه نمایش ماتریسی ۸ در ۸، به ۶۴ پین کنترلی نیاز داریم. این کار نه با میکروکنترلر مورد استفاده ما قابل انجام است، و نه مقیاس پذیری خوبی دارد و مشخصا صفحه نمایش‌های امروزی که میلیون ها پیکسل دارند نیاز به میلیون‌ها پین کنترلی ندارند. برای حل این مشکل از تکنیکی به نام مالتی‌پلکسینگ استفاده می‌شود. در این روش دیودهای نوری در ۸ ردیف و ۸ ستون قرار می‌گیرند.
\newline
\textcolor{red}{\begin{nas}سوال: \end{nas}}
در مورد روش مالتی‌پلکسینگ تحقیق کنید و مختصرا توضیح دهید. چگونه با این روش یک تصویر کامل به نمایش در می‌آید؟

\newline
\begin{nas}چیپ \lr{MAX7219}\end{nas}
\newline

چیپ \lr{MAX7219}
برای ما کار مالتی‌پلکسینگ و کنترل صفحه‌نمایش را انجام می‌دهد. نحوه ارتباط با این ماژول با استفاده از پروتوکل ارتباطی \lr{SPI} است. در این ارتباط باید دستورات مشخصی را برای چیپ بفرستیم و چیپ با توجه به دستورات ارسالی ما صفحه‌نمایش را کنترل می‌کند. در واقع این چیپ شامل تعدادی رجیستر کنترلی است که ما باید مقادیر آن را کنترل کنیم. برای کنترل هر کدام از این رجیسترها ابتدا در یک باید آدرس رجیستر فرستاده می‌شود و سپس در یک بایت دیگر مقدار ارسالی فرستاده می‌شود.

\newline
\textcolor{red}{\begin{nas}سوال: \end{nas}}
در ابتدای کار با این چیپ، باید تعداد \lr{Scan Line} ها برابر هشت قرار داده شود. در دیتاشیت این چیپ بگردید و آدرس رجیستر مربوطه، مقداری که باید به آن داده شود و همچنین قطعه کد آردوینو مربوط به تعیین این رجیستر را بنویسید.

\newline
\textcolor{red}{\begin{nas}سوال: \end{nas}}
در ابتدای کار با این چیپ، باید  
\lr{Test Mode}
غیر فعال شود. در دیتاشیت این چیپ بگردید و آدرس رجیستر مربوطه، مقداری که باید به آن داده شود و همچنین قطعه کد آردوینو مربوط به تعیین این رجیستر را بنویسید

\newline
\textcolor{red}{\begin{nas}سوال: \end{nas}}
روشن و خاموش بودن ال‌ای‌دی ها به صورت ردیفی کنترل می‌شود. برای کنترل هر ردیف ابتدا یک بایت ردیف مورد نظر و سپس یک بایت ال‌ای‌دی های روشن به صورت کدینگ باینری فرستاده می‌شود. در این کدینگ هر بیت یک معادل یک ال‌ای‌دی روشن در ردیف مورد نظر است. قطعه کد آردوینو مربوط به روشن کردن ال‌ای‌دی پنجم ردیف چهارم را بنویسید.

\subsection{شرح آزمایش}

\begin{enumerate}
    \item ماژول صفحه‌نمایش را به آردوینو متصل کنید. پین‌های \lr{VCC} و \lr{GND} به ترتیب به ۵ ولت و زمین وصل می‌شوند و پین‌های دیگر که مربوط به ارتباط \lr{SPI} هستند را می‌توانید از قسمت راهنمای پین‌های آردوینو در دستور کار مشاهده کنید. توجه کنید که به دلیل اینکه صفحه‌نمایش داده‌ای ارسال نمی‌کند پین \lr{MISO} نداریم و پین \lr{MOSI} نیز روی ماژول ممکن است به اسم \lr{DIN} نوشته شده باشد. پین چیپ سلکت را به یکی از پین های دیجیتال متصل کنید.
    \item کدی بنویسید که عبارت \lr{HELLO} را به صورتی که هر ثانیه یکی از کاراکترهای عبارت روی صفحه‌نمایش باشند نمایش دهد.
\end{enumerate}