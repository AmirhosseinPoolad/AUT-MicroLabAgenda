\section{کارگاه اسمبلی دوم}

\subsection{اهداف آزمایش}
\begin{itemize}
    \item آشنایی با نحوه نوشتن شرط و حلقه در زبان اسمبلی
    
\end{itemize}

\subsection{قطعات مورد نیاز}
\begin{itemize}
    \item کامپیوتر متصل به اینترنت و یا کامپیوتری که سیستم عامل لینوکس دارد
\end{itemize}

\subsection{مقدمه}

\paragraph{
یکی از ساختار هایی که در زبان های برنامه‌نویسی وجود دارد، اجرای شرطی و حلقه ها هستند. پیاده سازی این ساختار ها در زبان اسمبلی نیاز به تمرین دارد و در این کارگاه سعی می‌کنیم با این ساختار ها آشناتر شویم.
}

\subsection{دستور کارگاه}

\subsubsection{آشنایی با نوشتن شرط در زبان اسمبلی}
\begin{itemize}
    \item با استفاده از دستور \lr{B} می‌توانیم به یک قسمت دیگر از برنامه برویم. با استفاده از لیبل ها، برنامه‌ای دلخواه بنویسید که شامل ۵ دستور باشد، اما با گذاشتن دستور \lr{B} بعد از اجرای دستور دوم از دستور های سوم و چهارم پرش شود و دستور پنجم اجرا شود.
    \item حال بالای دستور پرش یک دستور جمع قرار دهید که نتیجه آن برابر با صفر شود.
    \item دستور پرش را به گونه‌ای تغییر دهید که فقط اگر نتیجه دستور قبل بزرگتر (نه بزرگتر‌ مساوی) از صفر بود پرش انجام شود. برنامه خود را اجرا کنید و نتیجه را مشاهده کنید.
    \item دستور جمعی که نوشته بودید را به گونه‌ای تغییر دهید که نتیجه برابر با صفر شود و دوباره نتیجه را مشاهده کنید.
\end{itemize}

\pagebreak

قطعه کد های زیر را مشاهده کنید که کد سی یک دستور شرطی و اسمبلی آن است که توسط کامپایلر ایجاد شده. توجه کنید که کد تولید شده بهینه نیست و می توان چند تا از دستور ها را حذف کرد.
\begin{latin}
\begin{lstlisting}
void compare(int a, int b) {
    if (a > b)
    {
        doIf();
    }
    else
    {
        doElse();
    }
}
\end{lstlisting}
\end{latin}

\begin{latin}
\begin{lstlisting}
        cmp     r0, r1
        ble     .ELSE_TRUE
        b       IF_TRUE
.IF_TRUE:
        bl      doIf()
        b       .POST_IF
.ELSE_TRUE:
        bl      doElse()
        b       .POST_IF
.POST_IF:
\end{lstlisting}
\end{latin}

به طور مشابه کدی بنویسید که دستور شرطی زیر را اجرا کند:
\begin{latin}
\begin{lstlisting}
    if(r0 <= r1) r0 = 0;
    else r0 = 1;
}
\end{lstlisting}
\end{latin}

\subsubsection{آشنایی با نوشتن حلقه در زبان اسمبلی}

\begin{itemize}
    \item با استفاده از دستور برنچ، یک حلقه بی‌نهایت ایجاد کنید که در آن دائما یکی به مقدار رجیستر صفر اضافه می‌شود.
    \item دستور برنچ را به گونه‌ای تغییر دهید که هنگامی که رجیستر صفر به مقدار ۱۰ رسید دیگر حلقه تکرار نشود.
    \item مقدار اولیه رجیستر صفر را قبل از اجرای حلقه به مقدار ۵ تغییر دهید.
    \item حلقه زیر را به زبان اسمبلی بنویسید
\end{itemize}

\begin{latin}
\begin{lstlisting}
for(r0 = 3, r0 < 15; r0 += 2)
{
    r1 += 1;
}
\end{lstlisting}
\end{latin}

\subsubsection{نوشتن یک برنامه کمی پیچیده‌تر اسمبلی}

می‌خواهیم برنامه‌ای که در کارگاه قبل نوشتیم را کمی پیشرفته تر کنیم. یک رشته دلخواه شامل حروف کوچک و بزرگ را در بالای برنامه خود قرار دهید و با استفاده از حلقه و شروط برنامه‌ای بنویسید که حروف کوچک رشته را به حروف بزرگ و حروف بزرگ رشته را به حروف کوچک تبدیل کند. اندازه رشته را می‌توانید به صورت هاردکود شده در برنامه داشته باشید، یا اینکه با استفا